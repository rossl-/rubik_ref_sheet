\documentclass[]{article}

\usepackage{titlesec}

\renewcommand\thesection{\Roman{section}}
\renewcommand\thesubsection{\thesection.\Alph{subsection}}
\titleformat{\section}[block]{\bfseries}{\thesection.}{1em}{}
\titleformat{\subsection}[block]{\hspace{2em}}{\thesubsection}{1em}{}

\newcommand{\Rubikbracket}[1]{$\left(\mbox{#1}\right)$}
\newcommand{\Rubikbrace}[1]{$\left\{\mbox{#1}\right\}$}

\makeatletter
% we use \prefix@<level> only if it is defined
\renewcommand{\@seccntformat}[1]{%
  \ifcsname prefix@#1\endcsname
  \csname prefix@#1\endcsname
  \else
  \csname the#1\endcsname\quad
  \fi}
% define \prefix@section
\newcommand\prefix@section{}
\newcommand\prefix@subsection{}
\makeatother

\author{}
\usepackage[margin=0.5in]{geometry}
\usepackage{tikz}
\usepackage{rubikcube,rubikrotation,rubikpatterns}
\usepackage{wrapfig}
\begin{document}

% ==========================================
\section{Rubik Notation}

\begin{wrapfigure}{r}{4cm}
  \vspace{-50pt}
  \begin{center}
    \RubikFaceUpAll{W}
    \RubikFaceFrontAll{O}
    \RubikFaceRightAll{G}
    \ShowCube{4cm}{0.5}{%
      \DrawRubikCubeRU%
      %% Right
      \draw[line width=1pt,color=blue,->] (3.5,2)-- (5.3, 2);
      \node(R) at (4.6, 2.5) [black]{\textbf{\textsf{R}}};
      \node(x) at (5.8, 2) [black]{\textbf{\textsf{x}}};
      %% Left
      \draw[line width=1pt,color=blue] (-0.2,2)-- (-1.5, 2);
      \node(L) at (-1.1, 2.5) [black]{\textbf{\textsf{L}}};
      %% Up
      \draw[line width=1pt,color=blue,->] (2, 3.5)-- (2, 5.5);
      \node(U) at (1.4, 4.7) [black]{\textbf{\textsf{U}}};
      \node(y) at (2, 6.1) [black]{\textbf{\textsf{y}}};
      %% Down
      \draw[line width=1pt,color=blue] (2, -0.2)-- (2, -1.5);
      \node(D) at (2.6, -1.1) [black]{\textbf{\textsf{D}}};
      %% Front
      \draw[line width=1pt,color=blue,->] (1.5, 1.5)-- (0, -1);
      \node(F) at (0.7, -0.7) [black]{\textbf{\textsf{F}}};
      \node(z) at (-0.3, -1.4) [black]{\textbf{\textsf{z}}};
      %% Back
      \draw[line width=1pt,color=blue] (3.2, 4.2)-- (4, 5.5);
      \node(B) at (4.4, 5) [black]{\textbf{\textsf{B}}};
    }
  \end{center}
  \caption{Rubik Notation}
\end{wrapfigure}

Rubik Notation includes the glyphs, letters, and symbols that make up
a single 90 degree rotation (or ``move'') of the Rubik Cube.\\
An attempt is made to limit move sequences to the following notation
(Up, Right, Left, Front):\\
\ShowSequence{}{\Rubik}{U} \ShowSequence{}{\Rubik}{Up}
\ShowSequence{}{\Rubik}{Um} \ShowSequence{}{\Rubik}{Ump} \ShowSequence{}{\Rubik}{Us}
\ShowSequence{}{\Rubik}{R}\ldots
\ShowSequence{}{\Rubik}{L}\ldots
\ShowSequence{}{\Rubik}{F}\ldots\\
There is a preference to move the whole cube to access Back \ShowSequence{}{\Rubik}{B}
and Down \ShowSequence{}{\Rubik}{D}.\\
As \ShowSequence{}{\Rubik}{y}$^2$ will rotate the whole cube 180 degrees,
moving the Back face to the Front.

\section{Rubik Solving}

\begin{wrapfigure}{r}{5cm}
  \vspace{20pt}
  \begin{center}
    \RubikCubeSolvedWY%
    \ShowCube{4cm}{0.4}{\DrawRubikCubeF}
  \end{center}
  \caption{Flat Cube (white opposed by yellow)}

  \begin{center}
    \RubikCubeSolvedWB%
    \ShowCube{4cm}{0.4}{\DrawRubikCubeF}
  \end{center}
  \caption{Flat Cube (white opposed by blue)}

  \begin{center}
    \RubikCubeSolvedWY%
    \RubikRotation{random,120}
    \ShowCube{4cm}{0.4}{\DrawRubikCubeF}
  \end{center}
  \caption{Random Cube}

\end{wrapfigure}

Make top white face, then \ShowSequence{}{\Rubik}{x}$^2$ to bring
yellow to the top:
\subsection{Edge Moves (use to make middle layer)}
\noindent%
(a) \newcommand{\edgemoveA}{U,R,Up,Rp,Up,Fp,U,F}%
\RubikCubeGreyAll{}%
\RubikFaceUp{X}{X}{X}%
            {X}{X}{X}%
            {X}{B}{X}%
\RubikSliceTopR{X}{O}{X}{X}{X}{X}
\ShowCube{1.6cm}{0.4}{\DrawRubikCubeRU}%
\RubikRotation{\edgemoveA}%
\quad
\ShowSequence{}{\Rubik}{\edgemoveA}%
\quad$\longrightarrow$\quad%
\ShowCube{1.6cm}{0.4}{\DrawRubikCubeRU}\\
\setlength{\parindent}{10ex}\par
(b) \newcommand{\edgemoveB}{Up,Lp,U,L,U,F,Up,Fp}%
\RubikCubeGreyAll{}%
\RubikFaceUp{X}{X}{X}%
            {X}{X}{X}%
            {X}{B}{X}%
\RubikSliceTopR{X}{O}{X}{X}{X}{X}
\ShowCube{1.6cm}{0.4}{\DrawRubikCubeLU}%
\RubikRotation{\edgemoveB}%
\quad
\ShowSequence{}{\Rubik}{\edgemoveB}%
\quad$\longrightarrow$\quad%
\ShowCube{1.6cm}{0.4}{\DrawRubikCubeLU}\\

\subsection{Corner Swaps (use to get corners in correct position)}
\noindent%
(a) \newcommand{\cornerswapA}{Lp,Up,L,F,U,Fp,Lp,U,L,U2}%
\RubikCubeGreyAll{}%
\RubikFaceUp{X}{X}{X}%
            {X}{X}{X}%
            {B}{X}{G}%
\RubikFaceLeft{X}{X}{Y}%
              {X}{X}{X}%
              {X}{X}{X}%
\RubikFaceRight{Y}{X}{X}%
               {X}{X}{X}%
               {X}{X}{X}%
\RubikFaceFront{O}{X}{O}%
               {X}{X}{X}%
               {X}{X}{X}%
\ShowCube{1.6cm}{0.4}{\DrawRubikCubeRU}%
\RubikRotation{\cornerswapA}%
\quad
\ShowSequence{}{\Rubik}{\SequenceShort}\ShowSequence{}{\Rubik}{U}$2$%
\quad$\longrightarrow$\quad%
\ShowCube{1.6cm}{0.4}{\DrawRubikCubeRU}\\
\setlength{\parindent}{10ex}\par
(b) \newcommand{\cornerswapB}{Lp,Up,L,F,U2,Fp,Lp,U,L}%
\RubikCubeGreyAll{}%
\RubikFaceUp{B}{X}{G}%
            {X}{X}{X}%
            {B}{X}{G}%
\RubikFaceLeft{B}{X}{B}%
              {X}{X}{X}%
              {X}{X}{X}%
\RubikFaceRight{G}{X}{G}%
               {X}{X}{X}%
               {X}{X}{X}%
\RubikFaceFront{B}{X}{G}%
               {X}{X}{X}%
               {X}{X}{X}%
\RubikFaceBack{G}{X}{B}%
              {X}{X}{X}%
              {X}{X}{X}%
\ShowCube{1.6cm}{0.4}{\DrawRubikCubeRU}%
\RubikRotation{\cornerswapB}%
\quad
\ShowSequence{}{\Rubik}{Lp,Up,L,F,U}$2$\ShowSequence{}{\Rubik}{Fp,Lp,U,L}%
\quad$\longrightarrow$\quad%
\ShowCube{1.6cm}{0.4}{\DrawRubikCubeRU}\\

\subsection{Corner Side-to-Top (use to flip sides up to top)}
\noindent%
(a) \newcommand{\cornersidetopA}{R,U,Rp,U,R,U2,Rp,U2}%
\ShowCube{1.6cm}{0.4}{%
  \DrawRubikLayerFace{X}{X}{X} {X}{X}{X} {Y}{X}{X}
  \DrawRubikLayerSideLTy{Y}
  \DrawRubikLayerSideRTx{Y}
  \DrawRubikLayerSideRBy{Y}
  \node(LB) at (0.5, 0.5) [black]{\small\textsf{X}};
}
\RubikRotation{\cornersidetopA}%
\quad
\ShowSequence{}{\Rubik}{R,U,Rp,U,R,U}$2$\ShowSequence{}{\Rubik}{Rp,U}$2$%
\quad$\longrightarrow$\quad%
\ShowCube{1.6cm}{0.4}{%
  \DrawRubikLayerFace{Y}{X}{Y} {X}{X}{X} {Y}{X}{Y}
  \node(LB) at (0.5, 0.5) [black]{\small\textsf{X}};
}\\
\setlength{\parindent}{10ex}\par
(b) \newcommand{\cornersidetopB}{Lp,Up,L,Up,Lp,U2,L,U2}%
\ShowCube{1.6cm}{0.4}{%
  \DrawRubikLayerFace{X}{X}{X} {X}{X}{X} {X}{X}{Y}
  \DrawRubikLayerSideLTx{Y}
  \DrawRubikLayerSideRTy{Y}
  \DrawRubikLayerSideLBy{Y}
  \node(LB) at (2.5, 0.5) [black]{\small\textsf{X}};
}
\RubikRotation{\cornersidetopB}%
\quad
\ShowSequence{}{\Rubik}{Lp,Up,L,Up,Lp,U}$2$\ShowSequence{}{\Rubik}{L,U}$2$%
\quad$\longrightarrow$\quad%
\ShowCube{1.6cm}{0.4}{%
  \DrawRubikLayerFace{Y}{X}{Y} {X}{X}{X} {Y}{X}{Y}
  \node(LB) at (2.5, 0.5) [black]{\small\textsf{X}};
}\\

\subsection{Edge Cycle (use to cycle edges, make sure corners are correct!)}
\noindent%
(a) \newcommand{\edgecycleA}{Rm,Up,Lm,U2,Rm,Up,Lm}%
\RubikCubeSolved%
\ShowCube{1.6cm}{0.4}{\DrawRubikCubeRU}%
\quad
\ShowCube{1.6cm}{0.4}{%
  \DrawFlatUpSide%
  \draw[thick,->,color=magenta] (2.5,1.5)-- (1.6, 2.4);
  \draw[thick,->,color=magenta] (1.5,2.5)-- (0.6, 1.6);
  \draw[thick,->] (0.5,1.4)-- (2.4, 1.4);
}%
\RubikRotation{\edgecycleA}%
\quad
\ShowSequence{}{\Rubik}{Rm,Up,Lm,U}$2$\ShowSequence{}{\Rubik}{Rm,Up,Lm}%
\quad$\longrightarrow$\quad%
\ShowCube{1.6cm}{0.4}{\DrawFlatUpSide}%\\
\vspace{5pt}
\setlength{\parindent}{10ex}\par
(b) \newcommand{\edgecycleB}{Rm,U,Lm,U2,Rm,U,Lm}%
\RubikCubeSolved%
\ShowCube{1.6cm}{0.4}{\DrawRubikCubeRU}%
\quad
\ShowCube{1.6cm}{0.4}{%
  \DrawFlatUpSide%
  \draw[thick,->,color=magenta] (1.5,2.5)-- (2.4, 1.6);
  \draw[thick,->,color=magenta] (0.5,1.5)-- (1.4, 2.4);
  \draw[thick,->] (2.5,1.4)-- (0.6, 1.4);
}%
\RubikRotation{\edgecycleB}%
\quad
\ShowSequence{}{\Rubik}{Rm,U,Lm,U}$2$\ShowSequence{}{\Rubik}{Rm,U,Lm}%
\quad$\longrightarrow$\quad%
\ShowCube{1.6cm}{0.4}{\DrawFlatUpSide}%\\

\subsection{Dedmore ``H'' Pattern}
\noindent%
(a) \newcommand{\dedmorehA}{Rm,U,Rm,U,Rm,U,Rm,U,Rm,Up,Rm,Up,Rm,Up,Rm,Up}%
\ShowCube{1.6cm}{0.4}{%
  \DrawRubikLayerFace{X}{X}{X} {G}{X}{B} {X}{X}{X}
  \DrawRubikLayerSideLMx{Y}
  \DrawRubikLayerSideRMx{Y}
}
\RubikRotation{\cornersidetopB}%
\quad
\Rubikbracket{\ShowSequence{}{\Rubik}{Rm,U}}$4$\Rubikbracket{\ShowSequence{}{\Rubik}{Rm,Up}}$4$%
\quad$\longrightarrow$\quad%
\ShowCube{1.6cm}{0.4}{%
  \DrawRubikLayerFace{X}{X}{X} {Y}{X}{Y} {X}{X}{X}
  \DrawRubikLayerSideLMx{G}
  \DrawRubikLayerSideRMx{B}
}\\
\setlength{\parindent}{10ex}\par
(b) \newcommand{\dedmorehB}{Rp,Um,R2,Um2,Rp,U2,R,Um2,R2,Ump,R,U2}%
\RubikRotation{\cornersidetopB}%
\quad
\ShowSequence{}{\Rubik}{Rp,Um,R}$2$\ShowSequence{}{\Rubik}{Um}$2$\ShowSequence{}{\Rubik}{Rp,U}$2$\ShowSequence{}{\Rubik}{R,Um}$2$\ShowSequence{}{\Rubik}{R}$2$\ShowSequence{}{\Rubik}{Ump,R,U}$2$%
\quad$\longrightarrow$\quad%
\\

\subsection{Dedmore ``Fish'' Pattern}
\noindent%
(a) \newcommand{\dedmorefishA}{F,R,Rm,U,Rm,U,Rm,U,Rm,U,Rm,Up,Rm,Up,Rm,Up,Rm,Up,Rp,Fp}%
\ShowCube{1.6cm}{0.4}{%
  \DrawRubikLayerFace{X}{X}{X} {G}{X}{X} {X}{B}{X}
  \DrawRubikLayerSideLMx{Y}
  \DrawRubikLayerSideMBy{Y}
}
\RubikRotation{\cornersidetopB}%
\quad
\Rubikbracket{\ShowSequence{}{\Rubik}{F,R}} + Dedmore ``H'' + \Rubikbracket{\ShowSequence{}{\Rubik}{Rp,Fp}}
\quad$\longrightarrow$\quad%
\ShowCube{1.6cm}{0.4}{%
  \DrawRubikLayerFace{X}{X}{X} {Y}{X}{X} {X}{Y}{X}
  \DrawRubikLayerSideLMx{G}
  \DrawRubikLayerSideMBy{B}
}\\


\section{Rubik Patterns}

A Rubik pattern is the configuration generated by a sequence of
rotations from some initial starting configuration (typically a ``solved'' configuration).\\

\setlength{\parindent}{0ex}\par
\RubikCubeSolved%
\newcommand{\MyPonsAsinorum}{Rs2,Us2,y,Rs2,yp}%
\RubikRotation{\MyPonsAsinorum}%
\quad\SequenceBraceB{PonsAsinorum}{%
  \ShowSequence{}{\Rubik}{Rs}$2$\ShowSequence{}{\Rubik}{Us}$2$\ShowSequence{}{\Rubik}{y,Rs}$2$\ShowSequence{}{\Rubik}{yp}%
}
\quad$\longrightarrow$\quad%
\ShowCube{1.6cm}{0.4}{\DrawRubikCubeRU}\\%

\setlength{\parindent}{0ex}\par
\RubikCubeSolved%
\RubikRotation{\Superflip}%
\quad\SequenceBraceB{Superflip}{%
  \Rubikbrace{\Rubikbracket{\Rubik{Rmp}\Rubik{Up}}4 \Rubik{yp} \Rubik{x}}3
}
\quad$\longrightarrow$\quad%
\ShowCube{1.6cm}{0.4}{\DrawRubikCubeRU}\\%

\setlength{\parindent}{0ex}\par
\RubikCubeSolved%
\RubikRotation{\CubeInCube}%
\quad\SequenceBraceB{\SequenceName}{%
  \ShowSequence{}{\Rubik}{\SequenceLong}%
}
%% \SequenceName\space\SequenceLong%
\quad$\longrightarrow$\quad%
\ShowCube{1.6cm}{0.4}{\DrawRubikCubeRU}\quad\\%

\setlength{\parindent}{0ex}\par
\RubikCubeSolved%
\RubikRotation{\Python}%
\quad\SequenceBraceB{\SequenceName}{%
  \ShowSequence{}{\Rubik}{\SequenceLong}%
}
%% \SequenceName\space\SequenceLong%
\quad$\longrightarrow$\quad%
\ShowCube{1.6cm}{0.4}{\DrawRubikCubeRU}\quad\\%

\setlength{\parindent}{0ex}\par
\RubikCubeSolved%
\RubikRotation{\SixSpot}%
\quad\SequenceBraceB{\SequenceName}{%
  \ShowSequence{}{\Rubik}{\SequenceLong}%
}
%% \SequenceName\space\SequenceLong%
\quad$\longrightarrow$\quad%
\ShowCube{1.6cm}{0.4}{\DrawRubikCubeRU}\quad\\%

\setlength{\parindent}{0ex}\par
\RubikCubeSolved%
\RubikRotation{\OrthogonalBars}%
\quad\SequenceBraceB{\SequenceName}{%
  \ShowSequence{}{\Rubik}{\SequenceLong}%
}
%% \SequenceName\space\SequenceLong%
\quad$\longrightarrow$\quad%
\ShowCube{1.6cm}{0.4}{\DrawRubikCubeRU}\quad\\%

\setlength{\parindent}{0ex}\par
\RubikCubeSolved%
\RubikRotation{\SixTs}%
\quad\SequenceBraceB{\SequenceName}{%
  \ShowSequence{}{\Rubik}{\SequenceLong}%
}
%% \SequenceName\space\SequenceLong%
\quad$\longrightarrow$\quad%
\ShowCube{1.6cm}{0.4}{\DrawRubikCubeRU}\quad\\%

\setlength{\parindent}{0ex}\par
\RubikCubeSolved%
\RubikRotation{\ExchangedRings}%
\quad\SequenceBraceB{\SequenceName}{%
  \ShowSequence{}{\Rubik}{\SequenceLong}%
}
%% \SequenceName\space\SequenceLong%
\quad$\longrightarrow$\quad%
\ShowCube{1.6cm}{0.4}{\DrawRubikCubeRU}\quad\\%

%%%\foreach\patSeq [count=\patSeqi] in
%%%{\CheckerboardsThree,\CheckerboardsSix,\Stripes,\CubeInCubeInCube,\ChristmasCross,\PlummersCross,\Anaconda,\Python,\BlackMamba,\GreenMamba,\FemaleRattlesnake,\MaleRattlesnake,\FemaleBoa,\MaleBoa,\FourSpot,\SixSpot,\OrthogonalBars,\SixTs,\SixTwoOne,\ExchangedPeaks,\TwoTwistedPeaks,\FourTwistedPeaks,\ExchangedChickenFeet,\TwistedChickenFeet,\ExchangedRings,\TwistedRings,\EdgeHexagonTwo,\EdgeHexagonThree,\TomParksPattern,\RonsCubeInCube,\TwistedDuckFeet,\ExchangedDuckFeet}
%%%{%
%%%  \setlength{\parindent}{0ex}\par
%%%  \RubikCubeSolved%
%%%  \RubikRotation{\patSeq}%
%%%  \quad\SequenceBraceB{\SequenceName}{%
%%%    \ShowSequence{}{\Rubik}{\SequenceLong}%
%%%  }
%%%  %%\SequenceName\space\SequenceLong%
%%%  \quad$\longrightarrow$\quad%
%%%  \ShowCube{1.6cm}{0.4}{\DrawRubikCubeRU}\quad\\%
%%%}


\end{document}

%%% Local Variables:
%%% mode: latex
%%% TeX-master: t
%%% LaTeX-command: "latex --shell-escape"
%%% End:
